\section{Configure PPP}
This chapter describes how to configure a PPP connection.
\subsection{Basic PPP Configuration}
\verb!R1(config)# interface Serial 0/0/0!
\verb!R1(config-if)# encapsulation ppp!
\subsection{Basic PPP Compression}
\verb!R1(config)# interface Serial 0/0/0!
\verb!R1(config-if)# encapsulation ppp!
\verb!R1(config-if)# compress predictor!
\subsection{Basic PPP Link Quality Control}
\verb!R1(config)# interface Serial 0/0/0!
\verb!R1(config-if)# encapsulation ppp!
\verb!R1(config-if)# ppp quality 80!
\subsection{Basic PPP Link Quality Control}
\verb!R1(config)# interface multilink 1!
\verb!R1(config-if)# interface Serial 0/0/0!
\verb!R1(config-if)# interface Serial 0/0/1!

\subsection{Basic PPP PAP Authentication}
\textbf{Note!} The first command is the expected username and password which R3 will send!

\verb!R1(config)# username R3 secret class!
\verb!R1(config)# interface s0/0/0!
\verb!R1(config-if)# ppp authentication pap!
\verb!R1(config-if)# ppp pap sent-username R1 password cisco!

\subsection{Basic PPP CHAP Authentication}
\textbf{Note!} As opposed of PAP. CHAP passwords need to be identical
\verb!R1(config)# hostname Router1!
\verb!Router1(config)# username Router 3 secret cisco!
\verb!Router1(config)# interface s0/0/0!
\verb!Router1(config-if)# ppp authentication chap!

\subsection{Troubleshoot PPP}
\verb!R1# debug ppp packet!
\verb!R1# debug ppp negotiation!
\verb!R1# debug ppp authentication!
\verb!R1# debug ppp error!

\subsection{Verifying PPP Connection}
\verb!R1# show interface serial 0/0/0! \\
\verb!R1# show ppp multilink!
