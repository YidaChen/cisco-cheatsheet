\section{Spanning Tree}
This chapter describes how to configure Spanning Tree.
\subsection{Verify Spanning tree configuration}
\begin{verbatim}
All.
S1#show spanning-tree
Per VLAN.
S1#show spanning-tree vlan 1
Discover layer 2 topology (if cdp is enabled).
S1#show cdp neighbours
\end{verbatim}
\subsection{Configure root bridge}
\begin{verbatim}
Method 0: Do nothing and let the root bridge be 
determined by the lowest MAC address.
Method 1: Set specific switch as (secondary) root bridge.
S1(config)#spanning-tree VLAN 1 root primary
or 
S1(config)#spanning-tree VLAN 1 root secondary
Method 2: Give priority numbers to all switches. 
Lowest becomes root bridge (needs to be a multiple of 4096).
S1(config)#spanning-tree VLAN 1 priority 24576
\end{verbatim}
\subsection{Rapid spanning tree mode}
\begin{verbatim}
Enable.
S1(config)#spanning-tree mode rapid-pvst
\end{verbatim}
\subsection{PortFast and BPDU guard for access ports}
\begin{verbatim}
Method 1: Per interface.
S1(config)#interface f0/1
S1(config-if)#spanning-tree portfast
S1(config-if)#spanning-tree bpduguard enable
Method 2: Enable globally for nontrunking interfaces.
S1(config)#spanning-tree portfast default
Enable bpduguard on portfast enabled ports.
S1(config)#spanning-tree portfast bpduguard default
\end{verbatim}
