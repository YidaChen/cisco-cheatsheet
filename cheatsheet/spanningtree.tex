\section{Spanning Tree}
This chapter describes how to configure Spanning Tree.
\subsection{Verify Spanning tree configuration}
All: \\
\verb!S1# show spanning-tree! \\
Per VLAN: \\
\verb!S1# show spanning-tree vlan 1! \\
Discover layer 2 topology (\textit{if cdp is enabled}): \\
\verb!S1# show cdp neighbours!\\

\subsection{Configure root bridge}
Method 0: Do nothing and let the root bridge be determined by the lowest MAC address. \\
Method 1: Set specific switch as (secondary) root bridge. \\
\verb!S1(config)# spanning-tree VLAN 1 root primary! \\
\textit{-or-} \\
\verb!S1(config)# spanning-tree VLAN 1 root secondary! \\
Method 2: Give priority numbers to all switches. \\
Lowest becomes root bridge (needs to be a multiple of 4096). \\
\verb!S1(config)# spanning-tree VLAN 1 priority 24576!

\subsection{Rapid spanning tree mode}
Enable: \\
\verb!S1(config)# spanning-tree mode rapid-pvst! \\

\subsection{PortFast and BPDU guard for access ports}
Method 1: Per interface:
\begin{verbatim}
S1(config)# interface f0/1
S1(config-if)# spanning-tree portfast
S1(config-if)# spanning-tree bpduguard enable
\end{verbatim}
Method 2: Enable globally for nontrunking interfaces:
\begin{verbatim}
S1(config)#spanning-tree portfast default
Enable bpduguard on portfast enabled ports.
S1(config)#spanning-tree portfast bpduguard default
\end{verbatim}
