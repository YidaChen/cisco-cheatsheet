\section{AAA}
This chapter describes how to configure AAAA.

\subsection{SSH}
After the default SSH configuration:
\begin{verbatim}
R1(config)#ip domain-name ccnasecurity.com
R1(config)#crypto key generate rsa
R1(config)#ip ssh version 2
\end{verbatim}

\subsection{RADIUS}
\begin{verbatim}
  R1(config)# aaa new-model
  R1(config)# radius-server host 192.168.3.2
  R1(config)# radius-server key radiuspa55
  R1(config)# aaa authentication login default
              group radius local
\end{verbatim}
\begin{textbox}{gray}{NOTE}
  With the last command the router/switch first looks at the RADIUS server.
  If the RADIUS server is not available he uses the local login database.
\end{textbox}
Console via RADIUS:
\begin{verbatim}
  R1(config)# line console 0
  R1(config-line)# login authentication default
\end{verbatim}

SSH via RADIUS:
\begin{verbatim}
  R1(config)# line vty 0 15
  R1(config-line)# login authentication default
  R1(config-line)# transport mode ssh
\end{verbatim}
