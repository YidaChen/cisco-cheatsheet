\section{OSPF}
This chapter describes how to configure OSPF.
\subsection{Single-Area OSPF}
\verb!R1(config)# interface GigabitEthernet0/0!
\verb!R1(config-if)# bandwidth 1000000! \\
\verb!R1(config-if)# exit! \\
\verb!R1(config)# router ospf 10! \\
\verb!R1(config-router)# router-id 1.1.1.1! \\
\verb!R1(config-router)# auto-cost reference-bandwidth 1000!
\verb!R1(config-router)# network 172.16.1.0 0.0.0.255 area 0!
\verb!R1(config-router)# passive-interface g0/0!

\subsection{Single-Area OSPFv3}
\verb!R1(config)# ipv6 router ospf 10!
\verb!R1(config-router)# router-id 1.1.1.1!
\verb!R1(config-router)# auto-cost reference-bandwidth 1000!
\verb!R1(config-if)# interface GigabitEthernet 0/0!
\verb!R1(config-if)# bandwidth 1000000!
\verb!R1(config-if)# ipv6 ospf 10 area 0!

\subsection{Verifying Single-Area OSPF}
\begin{tcolorbox}
\textbf{Note!} To verify Single-Area OSPFv3 please use the ipv6 command.
\end{tcolorbox}
\verb!R1# show ip ospf neighbor! \\
\verb!R1# show ip protocols! \\
\verb!R1# show ip ospf! \\
\verb!R1# show ip ospf interface! \\
\verb!R1# show ip ospf interface brief! \\

\section{Multi-Area OSPF}
\begin{tcolorbox}
\textbf{Note!} The same commands are used as for Single-Area OSPF, except there are more area's. Carefully look which device belong to which area.
\end{tcolorbox}
