\documentclass[10pt,landscape]{article}
\usepackage{multicol}
\usepackage{calc}
\usepackage{ifthen}
\usepackage[a4paper,landscape]{geometry}
\usepackage{hyperref}
\usepackage{tcolorbox}

% This sets page margins to .5 inch if using letter paper, and to 1cm
% if using A4 paper. (This probably isn't strictly necessary.)
% If using another size paper, use default 1cm margins.
\ifthenelse{\lengthtest { \paperwidth = 11in}}
	{ \geometry{top=.5in,left=.5in,right=.5in,bottom=.5in} }
	{\ifthenelse{ \lengthtest{ \paperwidth = 297mm}}
		{\geometry{top=1cm,left=1cm,right=1cm,bottom=1cm} }
		{\geometry{top=1cm,left=1cm,right=1cm,bottom=1cm} }
	}

% Turn off header and footer
\pagestyle{empty}


% Redefine section commands to use less space
\makeatletter
\renewcommand{\section}{\@startsection{section}{1}{0mm}%
                                {-1ex plus -.5ex minus -.2ex}%
                                {0.5ex plus .2ex}%x
                                {\normalfont\large\bfseries}}
\renewcommand{\subsection}{\@startsection{subsection}{2}{0mm}%
                                {-1explus -.5ex minus -.2ex}%
                                {0.5ex plus .2ex}%
                                {\normalfont\normalsize\bfseries}}
\renewcommand{\subsubsection}{\@startsection{subsubsection}{3}{0mm}%
                                {-1ex plus -.5ex minus -.2ex}%
                                {1ex plus .2ex}%
                                {\normalfont\small\bfseries}}
\makeatother

% Define BibTeX command
\def\BibTeX{{\rm B\kern-.05em{\sc i\kern-.025em b}\kern-.08em
    T\kern-.1667em\lower.7ex\hbox{E}\kern-.125emX}}

% Don't print section numbers
\setcounter{secnumdepth}{0}


\setlength{\parindent}{0pt}
\setlength{\parskip}{0pt plus 0.5ex}


% Colored boxes
% new tcolorbox environment
% #1: tcolorbox options
% #2: color
% #3: box title
\newtcolorbox{textbox}[3][]
{
  colframe = #2!25,
  colback  = #2!10,
  coltitle = #2!20!black,
  title    = #3,
  #1,
}


% -----------------------------------------------------------------------

\begin{document}

\raggedright
\footnotesize
\begin{multicols}{3}


% multicol parameters
% These lengths are set only within the two main columns
%\setlength{\columnseprule}{0.25pt}
\setlength{\premulticols}{1pt}
\setlength{\postmulticols}{1pt}
\setlength{\multicolsep}{1pt}
\setlength{\columnsep}{2pt}

\begin{center}
     \Large{Cisco Cheat Sheet} \\
\end{center}

% Basic Commands
\section{Basic Configuration}
\subsection{Initial Commands}
Name the device: \\
\verb!Router# configure terminal! \\
\verb!Router(config)# hostname [hostname]! \\

\smallskip

Configure a banner: \\
\verb!R1(config)# banner motd $Autorized Access Only$! \\

\smallskip

Save the Changes: \\
\verb!R1# copy running-config startup-config!

\smallskip

Configure Interface IPv4:
\verb!R1(config)# interface gigabitethernet 0/0! \\
\verb!R1(config-if)# description Link to LAN 1! \\
\verb!R1(config-if)# ip address 192.168.10.1 255.255.255.0! \\
\verb!R1(config-if)# no shutdown! \\
\textit{-or-} \\
\verb!R1(config)# interface serial 0/0/0! \\
\verb!R1(config-if)# description Link to R2! \\
\verb!R1(config-if)# ip address 209.165.200.225 255.255.255.252! \\
\verb!R1(config-if)# clock rate 128000! \\
\verb!R1(config-if)# no shutdown! \\


\subsection{Secure Management Access}
\verb!R1(config)# enable secret class! \\
\verb!R1(config)# line console 0! \\
\verb!R1(config-line)# password cisco! \\
\verb!R1(config-line)# login! \\
\verb!R1(config-line)# exit! \\
\verb!R1(config)# line vty 0 4! \textit{$\leftarrow$ depending on the number of VTYs!} \\
\verb!R1(config-line)# password cisco! \\
\verb!R1(config-line)# login! \\
\verb!R1(config-exit)# exit! \\
\verb!R1(config)# service password-encryption!


% VLAN
\subsection{VLAN}


% ACL, Access Control Lists
\section{Access Control Lists}
This chapter describes how to configure Access Control Lists (ACLs).
\begin{tcolorbox}
\textbf{Note!} Each ACL contains an implicit DENY at the end!
\end{tcolorbox}


% IPv6
\section{IPv6}
This chapter describes how to configure IPv6.

\subsection{IPv6 Autoconfiguration}
\begin{tcolorbox}
\textbf{Note!} Autoconfiguration requires te leas amount of configuration but makes it difficult to remember the IPv6 addresses.
This method uses the MAC address of the device to create an IPv6 address with the \verb!FE80::! prefix.
\end{tcolorbox}
Begin by configuring the router. Enter the interface configuration mode and enable IPv6 on the interface.
\begin{verbatim}
R1(config)# ipv6 unicast-routing
R1(config)# interface FastEthernet0/0
R1(config-if)# ipv6 enable
\end{verbatim}
Next, configure a link local address and a global unicast address on the interface. This example uses \verb!eui-64! to reduce the configuration.
\begin{verbatim}
R1(config-if)# ipv6 address autoconfig
R1(config-if)# ipv6 add 2000::/64 eui-64
R1(config-if)# no shutdown
\end{verbatim}
Verify the interface is \textit{up} and has two IPv6 addresses.
\begin{verbatim}
R1>show ipv6 interface brief
\end{verbatim}

\subsection{IPv6 Static}
Begin by configuring a static IPv6 address on the router
\begin{verbatim}
R1(config)# ipv6 unicast-routing
R1(config)# interface FastEthernet0/0
R1(config-if)# ipv6 enable
R1(config-if)# 2000::1/64
R1(config-if)# no shutdown
\end{verbatim}

\subsection{IPv6 Static Routing}
Configuration commands for its static routing are similar to IPv4.
\begin{verbatim}
R1(config)# ipv6 unicast-routing
R1(config)# ipv6 route 2000:2::/64 2001::20
\end{verbatim}

\subsection{IPv6 Dynamic Routing}
\begin{verbatim}
R1(config)# interface FastEthernet0/0
R1(config-if)# ipv6 address 2000:1::1/64
R1(config-if)# ipv6 rip Net1 enable
R1(config-if)# ipv6 enable
R1(config-if)# interface FastEthernet0/1
R1(config-if)# ipv6 address 2001::10/64
R1(config-if)# ipv6 rip Net1 enable
R1(config-if)# ipv6 enable
\end{verbatim}


% Spanning Tree
\section{Spanning Tree}
This chapter describes how to configure Spanning Tree.
\subsection{Verify Spanning tree configuration}
\begin{verbatim}
All.
S1#show spanning-tree
Per VLAN.
S1#show spanning-tree vlan 1
Discover layer 2 topology (if cdp is enabled).
S1#show cdp neighbours
\end{verbatim}
\subsection{Configure root bridge}
\begin{verbatim}
Method 0: Do nothing and let the root bridge be 
determined by the lowest MAC address.
Method 1: Set specific switch as (secondary) root bridge.
S1(config)#spanning-tree VLAN 1 root primary
or 
S1(config)#spanning-tree VLAN 1 root secondary
Method 2: Give priority numbers to all switches. 
Lowest becomes root bridge (needs to be a multiple of 4096).
S1(config)#spanning-tree VLAN 1 priority 24576
\end{verbatim}
\subsection{Rapid spanning tree mode}
\begin{verbatim}
Enable.
S1(config)#spanning-tree mode rapid-pvst
\end{verbatim}
\subsection{PortFast and BPDU guard for access ports}
\begin{verbatim}
Method 1: Per interface.
S1(config)#interface f0/1
S1(config-if)#spanning-tree portfast
S1(config-if)#spanning-tree bpduguard enable
Method 2: Enable globally for nontrunking interfaces.
S1(config)#spanning-tree portfast default
Enable bpduguard on portfast enabled ports.
S1(config)#spanning-tree portfast bpduguard default
\end{verbatim}



% Link Aggregation
\section{Link Aggregation}
This chapter describes how to configure port channels and to apply and configure the Link Aggregation Control Protocol (LACP).
\subsection{Configure Interfaces}
\begin{verbatim}
S1(config)# interface range fe0/1-2
S1(config-if-range)# shutdown
S1(config-if-range)# channel-group 1 mode active
S1(config-if-range)# exit
S1(config)# interface port-channel 1
S1(config-if)# switchport mode trunk
S1(config-if)# switchport trunk allowed vlan 1,2,20
\end{verbatim}

\subsection{Verify Link Aggregation}
\begin{verbatim}
S1# show interface port-channel1
S1# show etherchannel summary
S1# show etherchannel port-channel
S1# show interfaces f0/1 etherchannel
\end{verbatim}

More information about \href{https://www.cisco.com/c/en/us/td/docs/ios/12_2sb/feature/guide/gigeth.html}{Link Aggregation Control Protocol (LACP) (802.3ad) for Gigabit Interfaces}.


% OSPF
\section{OSPF}
This chapter describes how to configure OSPF.
\subsection{Single-Area OSPF}
\verb!R1(config)# interface GigabitEthernet0/0!
\verb!R1(config-if)# bandwidth 1000000! \\
\verb!R1(config-if)# exit! \\
\verb!R1(config)# router ospf 10! \\
\verb!R1(config-router)# router-id 1.1.1.1! \\
\verb!R1(config-router)# auto-cost reference-bandwidth 1000!
\verb!R1(config-router)# network 172.16.1.0 0.0.0.255 area 0!
\verb!R1(config-router)# passive-interface g0/0!

\subsection{Single-Area OSPFv3}
\verb!R1(config)# ipv6 router ospf 10!
\verb!R1(config-router)# router-id 1.1.1.1!
\verb!R1(config-router)# auto-cost reference-bandwidth 1000!
\verb!R1(config-if)# interface GigabitEthernet 0/0!
\verb!R1(config-if)# bandwidth 1000000!
\verb!R1(config-if)# ipv6 ospf 10 area 0!

\subsection{Verifying Single-Area OSPF}
\begin{tcolorbox}
\textbf{Note!} To verify Single-Area OSPFv3 please use the ipv6 command.
\end{tcolorbox}
\verb!R1# show ip ospf neighbor! \\
\verb!R1# show ip protocols! \\
\verb!R1# show ip ospf! \\
\verb!R1# show ip ospf interface! \\
\verb!R1# show ip ospf interface brief! \\

\section{Multi-Area OSPF}
\begin{tcolorbox}
\textbf{Note!} The same commands are used as for Single-Area OSPF, except there are more area's. Carefully look which device belong to which area.
\end{tcolorbox}


% PPP
\section{Configure PPP}
This chapter describes how to configure a PPP connection.
\subsection{Basic PPP Configuration}
\verb!R1(config)# interface Serial 0/0/0!
\verb!R1(config-if)# encapsulation ppp!
\subsection{Basic PPP Compression}
\verb!R1(config)# interface Serial 0/0/0!
\verb!R1(config-if)# encapsulation ppp!
\verb!R1(config-if)# compress predictor!
\subsection{Basic PPP Link Quality Control}
\verb!R1(config)# interface Serial 0/0/0!
\verb!R1(config-if)# encapsulation ppp!
\verb!R1(config-if)# ppp quality 80!
\subsection{Basic PPP Link Quality Control}
\verb!R1(config)# interface multilink 1!
\verb!R1(config-if)# interface Serial 0/0/0!
\verb!R1(config-if)# interface Serial 0/0/1!

\subsection{Basic PPP PAP Authentication}
\begin{tcolorbox}
\textbf{Note!} The first command is the expected username and password which R3 will send!

\verb!R1(config)# username R3 secret class!
\verb!R1(config)# interface s0/0/0!
\verb!R1(config-if)# ppp authentication pap!
\verb!R1(config-if)# ppp pap sent-username R1 password cisco!

\subsection{Basic PPP CHAP Authentication}
\begin{tcolorbox}
\textbf{Note!} As opposed of PAP. CHAP passwords need to be identical
\verb!R1(config)# hostname Router1!
\verb!Router1(config)# username Router 3 secret cisco!
\verb!Router1(config)# interface s0/0/0!
\verb!Router1(config-if)# ppp authentication chap!

\subsection{Troubleshoot PPP}
\verb!R1# debug ppp packet!
\verb!R1# debug ppp negotiation!
\verb!R1# debug ppp authentication!
\verb!R1# debug ppp error!

\subsection{Verifying PPP Connection}
\verb!R1# show interface serial 0/0/0! \\
\verb!R1# show ppp multilink!


% Security
\section{Security}
This chapter explains how to secure devices 
\subsection{Commands to increase Acces Security}
\begin{verbatim}
R1(config)# security paswords min-length 10 
R1(config)# service password-encryption
R1(config)# line vty 0 4 
R1(config)# exec-timeout 3 30
R1(config)# line console 0 
R1(config)# exec-timeout 3 30
\end{verbatim}

\subsection{Enable Stronger Password Encryption}
\begin{tcolorbox}
\textbf{Note!}There are two methods of enabling a stronger password hash. The first one is when you already have a hash of the encrypted password. The second one is if you want to enter a password. The Second method does not work in Packet Tracer
\end{tcolorbox}
\begin{verbatim}
First Methode
R1(config)# enable secret 9 HZWdzLHwhPtZ3UD9OlUDSGvBy.m8Tf9vCGDJRcYy8zIMbyRJgtxgRkwzY \\

Second Method
R1(config)# enable algorithm-type scrypt secret cisco 
\end{verbatim}

\subsection{Password Encryption for username secret}
\begin{verbatim}
R1(config)# username Bob algorithm-type scrypt secret cisco
\end{verbatim}


\subsection{Configure Secure Line Acces}
\begin{verbatim}
R1(config)# username Bob algorithm-type scrypt secret cisco
\end{verbatim}




% Additional Resources
\section{Additional Resources}
Additional resources for more information about Cisco configuration.
\begin{description}
  \item [Cisco DocWiki] \url{http://docwiki.cisco.com/wiki/Main_Page}
\end{description}


% AAA
\section{AAA}
This chapter describes how to configure AAAA.

\subsection{SSH}
After the default SSH configuration:
\begin{verbatim}
R1(config)#ip domain-name ccnasecurity.com
R1(config)#crypto key generate rsa
R1(config)#ip ssh version 2
\end{verbatim}

\subsection{RADIUS}
\begin{verbatim}
  R1(config)# aaa new-model
  R1(config)# radius-server host 192.168.3.2
  R1(config)# radius-server key radiuspa55
  R1(config)# aaa authentication login default
              group radius local
\end{verbatim}
\begin{textbox}{gray}{NOTE}
  With the last command the router/switch first looks at the RADIUS server.
  If the RADIUS server is not available he uses the local login database.
\end{textbox}
Console via RADIUS:
\begin{verbatim}
  R1(config)# line console 0
  R1(config-line)# login authentication default
\end{verbatim}

SSH via RADIUS:
\begin{verbatim}
  R1(config)# line vty 0 15
  R1(config-line)# login authentication default
  R1(config-line)# transport mode ssh
\end{verbatim}


% Syslog
\section{Syslog}
This chapter describes how to configure Syslog.
\begin{verbatim}
  R1(config)# logging 192.168.1.3
  R1(config)# logging trap 4
  R1(config)# logging source-interface g0/0
  R1(config)# service timestamps log datetime msec
\end{verbatim}


% NTP
\section{NTP}
This chapter describes how to configure NTP.
\begin{verbatim}
R1(config)# ntp server 64.103.224.2
R1(config)# service timestamps log datetime msec
\end{verbatim}



\rule{0.3\linewidth}{0.25pt}
\scriptsize

\href{https://github.com/roaldnefs/cisco-cheatsheet}{https://github.com/roaldnefs/cisco-cheatsheet}


\end{multicols}
\end{document}
