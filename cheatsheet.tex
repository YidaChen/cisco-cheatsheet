\documentclass[10pt,landscape]{article}
\usepackage{multicol}
\usepackage{calc}
\usepackage{ifthen}
\usepackage[a4paper,landscape]{geometry}
\usepackage{hyperref}
\usepackage{tcolorbox}

% This sets page margins to .5 inch if using letter paper, and to 1cm
% if using A4 paper. (This probably isn't strictly necessary.)
% If using another size paper, use default 1cm margins.
\ifthenelse{\lengthtest { \paperwidth = 11in}}
	{ \geometry{top=.5in,left=.5in,right=.5in,bottom=.5in} }
	{\ifthenelse{ \lengthtest{ \paperwidth = 297mm}}
		{\geometry{top=1cm,left=1cm,right=1cm,bottom=1cm} }
		{\geometry{top=1cm,left=1cm,right=1cm,bottom=1cm} }
	}

% Turn off header and footer
\pagestyle{empty}


% Redefine section commands to use less space
\makeatletter
\renewcommand{\section}{\@startsection{section}{1}{0mm}%
                                {-1ex plus -.5ex minus -.2ex}%
                                {0.5ex plus .2ex}%x
                                {\normalfont\large\bfseries}}
\renewcommand{\subsection}{\@startsection{subsection}{2}{0mm}%
                                {-1explus -.5ex minus -.2ex}%
                                {0.5ex plus .2ex}%
                                {\normalfont\normalsize\bfseries}}
\renewcommand{\subsubsection}{\@startsection{subsubsection}{3}{0mm}%
                                {-1ex plus -.5ex minus -.2ex}%
                                {1ex plus .2ex}%
                                {\normalfont\small\bfseries}}
\makeatother

% Define BibTeX command
\def\BibTeX{{\rm B\kern-.05em{\sc i\kern-.025em b}\kern-.08em
    T\kern-.1667em\lower.7ex\hbox{E}\kern-.125emX}}

% Don't print section numbers
\setcounter{secnumdepth}{0}


\setlength{\parindent}{0pt}
\setlength{\parskip}{0pt plus 0.5ex}


% -----------------------------------------------------------------------

\begin{document}

\raggedright
\footnotesize
\begin{multicols}{3}


% multicol parameters
% These lengths are set only within the two main columns
%\setlength{\columnseprule}{0.25pt}
\setlength{\premulticols}{1pt}
\setlength{\postmulticols}{1pt}
\setlength{\multicolsep}{1pt}
\setlength{\columnsep}{2pt}

\begin{center}
     \Large{Cisco Cheat Sheet} \\
\end{center}

\section{Basic Configuration}
\subsection{Initial Commands}
Name the device: \\
\verb!Router# configure terminal! \\
\verb!Router(config)# hostname [hostname]! \\

\smallskip

Configure a banner: \\
\verb!R1(config)# banner motd $Autorized Access Only$! \\

\smallskip

Save the Changes: \\
\verb!R1# copy running-config startup-config!

\smallskip

Configure Interface IPv4:
\verb!R1(config)# interface gigabitethernet 0/0! \\
\verb!R1(config-if)# description Link to LAN 1! \\
\verb!R1(config-if)# ip address 192.168.10.1 255.255.255.0! \\
\verb!R1(config-if)# no shutdown! \\
\textit{-or-} \\
\verb!R1(config)# interface serial 0/0/0! \\
\verb!R1(config-if)# description Link to R2! \\
\verb!R1(config-if)# ip address 209.165.200.225 255.255.255.252! \\
\verb!R1(config-if)# clock rate 128000! \\
\verb!R1(config-if)# no shutdown! \\


\subsection{Secure Management Access}
\verb!R1(config)# enable secret class! \\
\verb!R1(config)# line console 0! \\
\verb!R1(config-line)# password cisco! \\
\verb!R1(config-line)# login! \\
\verb!R1(config-line)# exit! \\
\verb!R1(config)# line vty 0 4! \textit{$\leftarrow$ depending on the number of VTYs!} \\
\verb!R1(config-line)# password cisco! \\
\verb!R1(config-line)# login! \\
\verb!R1(config-exit)# exit! \\
\verb!R1(config)# service password-encryption!


\subsection{VLAN}

\subsection{Access Control Lists}
This chapter describes how to configure Access Control Lists (ACLs).
\begin{tcolorbox}
\textbf{Note!} Each ACL contains an implicit DENY at the end!
\end{tcolorbox}


\section{Spanning Tree}
This chapter describes how to configure Spanning Tree.

\section{Link Aggregation}
This chapter describes how to configure port channels and to apply and configure the Link Aggregation Control Protocol (LACP).
\subsection{Configure Interfaces}
\verb!s1(config)# interface range fe0/1-2!
\verb!s1(config-if-range)# shutdown!
\verb!s1(config-if-range)# channel-group 1 mode active!
\verb!s1(config-if-range)# exit!
\verb!s1(config)# interface port-channel 1!
\verb!s1(config-if)# switchport mode trunk!
\verb!s1(config-if)# switchport trunk allowed vlan 1,2,20!

\subsection{Verify Link Aggregation}
\verb!s1# show interface port-channel1!
\verb!s1# show etherchannel summary!
\verb!s1# show etherchannel port-channel!
\verb!s1# show interfaces f0/1 etherchannel!

More information about \href{https://www.cisco.com/c/en/us/td/docs/ios/12_2sb/feature/guide/gigeth.html}{Link Aggregation Control Protocol (LACP) (802.3ad) for Gigabit Interfaces}.

\section{OSPF}
This chapter describes how to configure OSPF.
\subsection{Single-Area OSPF}
\begin{tcolorbox}
\textbf{Note!} The same commands are used for Multi-Area OSPF, except there are more area's. Carefully look which device belong to which area.
\end{tcolorbox}
\verb!R1(config)# interface GigabitEthernet0/0!
\verb!R1(config-if)# bandwidth 1000000! \\
\verb!R1(config-if)# exit! \\
\verb!R1(config)# router ospf 10! \\
\verb!R1(config-router)# router-id 1.1.1.1! \\
\verb!R1(config-router)# auto-cost reference-bandwidth 1000!
\verb!R1(config-router)# network 172.16.1.0 0.0.0.255 area 0!
\verb!R1(config-router)# passive-interface g0/0!

\subsection{Single-Area OSPFv3}
\verb!R1(config)# ipv6 router ospf 10!
\verb!R1(config-router)# router-id 1.1.1.1!
\verb!R1(config-router)# auto-cost reference-bandwidth 1000!
\verb!R1(config-if)# interface GigabitEthernet 0/0!
\verb!R1(config-if)# bandwidth 1000000!
\verb!R1(config-if)# ipv6 ospf 10 area 0!

\subsection{Verifying Single-Area OSPF}
\begin{tcolorbox}
\textbf{Note!} To verify Single-Area OSPFv3 please use the ipv6 command.
\end{tcolorbox}
\verb!R1# show ip ospf neighbor! \\
\verb!R1# show ip protocols! \\
\verb!R1# show ip ospf! \\
\verb!R1# show ip ospf interface! \\
\verb!R1# show ip ospf interface brief! \\

%\section{Multi-Area OSPF}

\section{Configure PPP}
This chapter describes how to configure a PPP connection.
\subsection{Basic PPP Configuration}
\verb!R1(config)# interface Serial 0/0/0!
\verb!R1(config-if)# encapsulation ppp!
\subsection{Basic PPP Compression}
\verb!R1(config)# interface Serial 0/0/0!
\verb!R1(config-if)# encapsulation ppp!
\verb!R1(config-if)# compress predictor!
\subsection{Basic PPP Link Quality Control}
\verb!R1(config)# interface Serial 0/0/0!
\verb!R1(config-if)# encapsulation ppp!
\verb!R1(config-if)# ppp quality 80!
\subsection{Basic PPP Link Quality Control}
\verb!R1(config)# interface multilink 1!
\verb!R1(config-if)# interface Serial 0/0/0!
\verb!R1(config-if)# interface Serial 0/0/1!

\subsection{Basic PPP PAP Authentication}
\textit{Note: The first command is the expected username and password which R3 will send!}

\verb!R1(config)# username R3 secret class!
\verb!R1(config)# interface s0/0/0!
\verb!R1(config-if)# ppp authentication pap!
\verb!R1(config-if)# ppp pap sent-username R1 password cisco!

\subsection{Basic PPP CHAP Authentication}
\textit{Note: As opposed of PAP. CHAP passwords need to be identical}
\verb!R1(config)# hostname Router1!
\verb!Router1(config)# username Router 3 secret cisco!
\verb!Router1(config)# interface s0/0/0!
\verb!Router1(config-if)# ppp authentication chap!


\verb!R1(config)# username R3 secret class!
\verb!R1(config)# interface s0/0/0!
\verb!R1(config-if)# ppp authentication pap!
\verb!R1(config-if)# ppp pap sent-username R1 password cisco!
\verb! !


\subsection{Verifying PPP Connection}
\verb!R1# show interface serial 0/0/0! \\
\verb!R1# show ppp multilink!


\rule{0.3\linewidth}{0.25pt}
\scriptsize

\href{https://github.com/roaldnefs/cisco-cheatsheet}{https://github.com/roaldnefs/cisco-cheatsheet}


\end{multicols}
\end{document}
